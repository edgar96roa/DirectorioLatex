%=========================================================
\chapter{Modelo dinámico}
\label{cap:ModeloDinamico}

Este capítulo describe en modelo dinámico del sistema. en el se detallan todos los escenarios de ejecución del sistema asi como el comportamiento que tendra el sistema a lo largo de tiempo se uso.
%---------------------------------------------------------
%\section{Maquina de estados}

%---------------------------------------------------------
\clearpage
\section{Modelo de Maquinas de Estado}
\begin{figure}[htbp!]
    \centering
    \includegraphics[width=0.8\textwidth]{Capitulo4/edo_especta.jpg}
    \caption{Diagrama de estados del estatus de espectacular}
    \label{fig:my_label}
\end{figure}
\textbf{ins-estru:} el estatus del espectacular pasa a \textbf{Disponible} cuando se hace la instalación exitosa de la estructura del espectacular.
\\
\textbf{inic[contrato]:} Cuando el PIMDE hace la instalación de la lona espectacular, el estatus pasa a \textbf{No disponible} cuando hay un contrato. 
\\
\textbf{fin[Contrato]:} cuando se acaba el contrato de la renta de la estructura del espectacular, el estatus pasa a \textbf{Disponible}.
\\
\textbf{fecha-progra:} el estatus del espectacular pasa a \textbf{En mantenimiento} cuando la fecha programada para su mantenimiento esta próxima.
\\
\textbf{reg-positivo:} El estatus del espectacular pasa a \textbf{No disponible} o \textbf{Disponible} cuando hay un registro positivo del mantenimiento del mismo.
\\
\textbf{reg-negativo:} cuando hay un registro negativo del mantenimiento del espectacular se queda ahí mismo.
\textbf{inst-defe} Al momento de hacer un mantenimiento del espectacular y el PIMDE se percata que la estructura no soporta otro mantenimiento ya sea correctivo o preventivo, notifica al Gerente de Infraestructura para que cambie el estatus a \textbf{Baja}
\textbf{term[Contrato]:} Cuando se termina el contrato de renta de espacio con el arrendatario cambia el estatus del espectacular a \textbf{Baja}
\clearpage

\begin{figure}[htbp!]
    \centering
    \includegraphics[width=0.8\textwidth]{Capitulo4/edo_servicio.jpg}
    \caption{Diagrama de estados del estatus del servicio}
    \label{fig:my_label}
\end{figure}
\textbf{ini-esta:} 
el estatus del servicio que se realiza en la estructura del espectacular por defecto se encuentra en \textbf{No realizado}.
\\
\textbf{asig-serv:}
el estatus de la estructura del espectacular pasa a \textbf{En proceso}, cuando el Gerente de Infraestructura asigna al PIMDE un servicio y este se encuentra en el lugar en donde se tiene que realizar el servicio.
\\
\textbf{si\_Realizado:}
el estatus de la estructura del espectacular pasa a \textbf{Realizado}, cuando el PIMDE realizó un servicio exitoso sobre la estructura del espectacular.
\\
\textbf{no\_Realizado:}
el estatus de la estructura del espectacular pasa a \textbf{No realizado}, cuando el PIMDE no pudo realizar un servicio exitoso sobre la estructura del espectacular.
\clearpage

\section{Modelo de Casos de Uso}

\begin{figure}[htbp!]
    \centering
    \includegraphics[angle=90, width=0.80\textwidth]{Capitulo4/ModeloGeneralCU.jpg}
    \caption{Diagrama General de Casos de uso}
    \label{fig:my_label}
\end{figure}

% CASOS DE USO
\input{cu/CU00}
\input{cu/CU01}
\input{cu/CU02}
\input{cu/CU03}
\input{cu/CU04}
\input{cu/CU05}
\input{cu/CU06}
\input{cu/CU07}
%\input{cu/CU08}
%\input{cu/CU09}
%\input{cu/CU10}
\input{cu/CU11}
\input{cu/CU12}
\input{cu/CU13}
\input{cu/CU14}
\input{cu/CU15}
\input{cu/CU16}
%\input{cu/CU17}
%\input{cu/CU18}
%\input{cu/CU19}
%\input{cu/CU20}
\input{cu/CU21}
\input{cu/CU22}