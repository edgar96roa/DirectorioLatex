%=========================================================
\chapter{Diseño de Información}
\label{cap:DiseniodeInformacion}
\section{Modelo Relacional}
hace referencia a:
\hyperlink{cap:infoydats}{Información y datos}
\section{Diccionario de Datos}
hace referencia a:
\hyperlink{cap:metadatos}{Especificación de atributos}
\section{Modelo de Acceso a datos}
En esta sección se describen como serán las consultas que se usaran en los diagramas de secuencia, con la etiqueta especificada de la siguiente manera \textbf{DS-M: SQL Query
\#N}
\subsection{DS-M: SQL Query \#1}
Esta es la consulta que nos permite obtener la información de los usuario que quieran ingresar al sistema, se presentan dos consultas una para los trabajadores de la empresa y otra para los clientes.    
\begin{verbatim}
CREATE OR REPLACE VIEW loginUsers AS
    SELECT e.email, e.password, p.nombrePuesto
    FROM Empleado e, puesto p
    where p.idPuesto=e.idPuesto
    UNION
    SELECT email, password, "Cliente"
    FROM Cliente;
    
SELECT password into pswd
FROM loginUsers
WHERE email = 'email';
\end{verbatim}


\subsection{DS-M: SQL Query \#2}
Esta es la consulta que nos permite cambiar la contraseña de algun usuario que quiera ingresar a nuestro sistema.
\begin{verbatim}
DELIMITER #
CREATE PROCEDURE cambiocontraseña(IN emaill VARCHAR(45),IN passwordd VARCHAR(45))
BEGIN
SELECT email, password
FROM Cliente
WHERE password LIKE CONCAT (passwordd,"%");
	
UPDATE Cliente SET password=passwordd
WHERE email=emaill;
	
SELECT email, password
FROM Cliente
WHERE password LIKE CONCAT (passwordd,"%");
END #
\c 
DELIMITER ;
CALL cambiocontraseña("email","password");
\end{verbatim}


\subsection{DS-M: SQL Query \#3}
Esta consulta nos permite dar de alta un nuevo espectacular en el sistema para después realizar una renta sobre el mismo, siempre y cuando el id del espectacular no se halla registrado con anterioridad.
\begin{verbatim}
DELIMITER #
CREATE PROCEDURE altaespectacular(IN id INT, IN anc DOUBLE, IN lar DOUBLE, IN impaper INT, INT idfoto INT

, IN imagennn BLOB, IN idtipespe INT, IN descr VARCHAR(45), IN viiaa VARCHAR(45), IN ilumi VARCHAR(2)

, IN idubi INT, IN edo VARCHAR(45), IN delega VARCHAR(60), IN cpp NUMBER(5), IN colo VARCHAR(45)

, IN calleee VARCHAR(45), IN numext VARCHAR(10), IN idz INT)
BEGIN
INSERT INTO Espectacular VALUES (id,anc,lar,impaper,"","",idtipespe,idubi,"","");
INSERT INTO Galeria VALUES (id,idfoto);
INSERT INTO Fotografia VALUES (idfoto,imagennn);
INSERT INTO TipoEspectacular VALUES (idtipespe,descr,"",ilumi,viiaa);
INSERT INTO UbicacionEspectacular VALUES (idubi,"","",edo,delega,cpp,colo,calleee,numext,idz);
	
END #
\c 
DELIMITER ;
CALL altaespectacular("id", "anc", "lar", "impaper", "idfoto", "imagennn", "idtipespe", "descr", "viiaa",

"ilumi" , "idubi", "edo", "delega", "cpp", "colo", "calleee", "numext", "idz");
\end{verbatim}
   
   
\subsection{DS-M: SQL Query \#4}
En esta consulta nos permite hacer una selección de especifica acerca de un espectacular, la cual nos trae como atributos lo que se especifican a continuación en la query.
\begin{verbatim}
SELECT e.idEspectacular, et.descripcionDeEstado, te.descripcion, te.via, te.descripcion, ue.ancho,

ue.largo, ue.calle, ue.numeroExterior, ue.colonia, ue.cp, ue.delegacion, e.tamaño, e.precio
FROM estado et, espectacular e, TipoEspectacular te, UbicacionEspectacular ue
WHERE et.idestado=e.idestado
AND e.idTipoEspectacular=te.idTipoEspectacular
AND e.idUbicacion=ue.idUbicacion
AND e.idEspectacular="E04";
\end{verbatim}


\subsection{DS-M: SQL Query \#5}
En esta consulta nos permite hacer una selección de todos los espectaculares que tiene asociado un cliente, la cual nos trae como atributos lo que se especifican a continuación en la query.
\begin{verbatim}
SELECT e.idEspectacular, et.descripcionDeEstado, te.descripcion, te.via, te.descripcion, ue.ancho,

ue.largo, ue.calle, ue.numeroExterior, ue.colonia, ue.cp, ue.delegacion, e.tamaño, e.precio
FROM estado et, espectacular e, TipoEspectacular te, UbicacionEspectacular ue
WHERE et.idestado=e.idestado
AND e.idTipoEspectacular=te.idTipoEspectacular
AND e.idUbicacion=ue.idUbicacion
\end{verbatim}

\subsection{DS-M: SQL Query \#6}
En esta consulta nos permite dar la alta de un nuevo cliente a nuestro sistema para que el pueda hacer uso del mismo.
\begin{verbatim}
DELIMITER #
CREATE PROCEDURE altacliente(in idclie int, in idmora int, in idfisi, in rfcc varchar(13),

in namefisi varchar(60), in namemora varchar(60), in apmat varchar(60), in appat varchar(60),

in emaill varchar(45), in numeroTel varchar(45), in pass varchar(45), in calleeo varchar(60),

in delegac varchar(60), in cppp varchar(5), in colon varchar(45))
BEGIN
INSERT INTO TelefonoClienteFisico VALUES(idfisi,numeroTel,"");
INSERT INTO TelefonoClienteMoral VALUES(idmora,numeroTel,"","");
INSERT INTO ClientePersonaFisica VALUES(idfisi,namefisi,apmat,appat,idclie,rfcc);
INSERT INTO ClientePersonaMoral VALUES(idmora,namemora,"",idclie,rfcc);
INSERT INTO Cliente VALUES(idclie,"",emaill,pass,calleeo,delegac,cppp,colon,"",rfcc);
END #
DELIMITER ;
CALL altacliente(idclie,idmora,idfisi,rfcc,namefisi,namemora,apmat,appat,emaill,numeroTel,pass,calleeo,

delegac,cppp,colon);
\end{verbatim}


\subsection{DS-M: SQL Query \#7}
En esta consulta nos permite hacer una selección de los clientes que están registrados en el sistema, esta selección es de manera especifica, es decir nos trae un solo cliente.

Los atributos que nos trae esta consulta se muestran a continuación.
\begin{verbatim}
SELECT c.*, cpf.*, cpm.idClientePersonaMoral, cpm.razonSocial, cpm.idCliente, cpm.RFC, tcf.*,tcm.*
FROM TelefonoClienteFisico tcf, TelefonoClienteMoral tcm, ClientePersonaFisica cpf,

ClientePersonaMoral cpm, Cliente c
WHERE ((c.idCliente=cpf.idCliente AND cpf.idClientePersonaFisica=tcf.idClientePersonaFisica 

AND c.usuario="user") OR 
	  (c.idCliente=cpm.idCliente AND cpm.idClientePersonaMoral=tcm.idClienteMoral 
	  
	  AND c.usuario="user"));
\end{verbatim}



\subsection{DS-M: SQL Query \#8}
En esta consulta nos permite hacer una selección de todos los clientes que están registrados en el sistema, esta selección es de manera general, es decir nos trae a todos lo clientes que están en el sistema.

Los atributos que nos trae esta consulta se muestran a continuación.
\begin{verbatim}
SELECT c.*, cpf.*, cpm.idClientePersonaMoral, cpm.razonSocial, cpm.idCliente, cpm.RFC, tcf.*,tcm.*
FROM TelefonoClienteFisico tcf, TelefonoClienteMoral tcm, ClientePersonaFisica cpf,

ClientePersonaMoral cpm, Cliente c
WHERE ((c.idCliente=cpf.idCliente AND cpf.idClientePersonaFisica=tcf.idClientePersonaFisica) OR 
	  (c.idCliente=cpm.idCliente AND cpm.idClientePersonaMoral=tcm.idClienteMoral);
\end{verbatim}


\subsection{DS-M: SQL Query \#9}
En esta consulta nos permite hacer una proyección para la cotización de espectaculares de manera general, es decir no va a arrojar todos los espectaculares que cumplan con el criterio de búsqueda.
Los atributos que no arroja esta consulta se muestran a continuación.
\begin{verbatim}
SELECT c.idCliente, e.ImpactoEnPersonas, tc.tipoCampaña, te.via, e.alto, e.ancho, z.nombreZona,

ue.Delegacion, e.precio, te.descripcion
FROM Cliente c, Contrato co, DetalleDeContrato ddc, Espectacular e, TipoCampaña tc, TipoEspectacular, 

UbicacionEspectacular ue, Zona z
WHERE c.idCliente=co.idCliente
AND co.idTipoCampaña=tc.idTipoCampaña
AND co.idContrato=ddc.idContrato
AND ddc.idEspectacular=e.idEspectacular
AND e.idTipoEspectacular=te.idTipoEspectacular
AND e.idUbicacion=ue.idUbicacion
AND ue.idZona=z.idZona
AND e.ImpactoEnPersonas=100000;
\end{verbatim}


\subsection{DS-M: SQL Query \#10}
En esta consulta nos permite recuperar la contraseña de algún cliente o usuario que desee recuperar la misma. El campo que tendrá que introducir el cliente o usuario seria su email para que le sea devuelta la contraseña.
\begin{verbatim}
DELIMITER #
CREATE PROCEDURE reccontraseña(in emaill varchar(45))
BEGIN
select email, password
from cliente
where email like concat(emaill,"%");
END #
\c 
DELIMITER ;
CALL reccontraseña("christian30394@gmail.com");
\end{verbatim}


\subsection{DS-M: SQL Query \#11}
En esta consulta nos permite hacer la selección de los trabajos que ha realizado un pimde a los espectaculares, esto es de manera general, es decir nos va a seleccionar todos los trabajos que ha realizado.
\begin{verbatim}
DELIMITER #
CREATE PROCEDURE trabajos(in numemp int)
BEGIN
select e.nombre, s.tipoDeServicio, es.idEspectacular
from empleado e, realiza r, servicio s, espectacular es
where e.numeroEmpleado=r.numeroEmpleado
and r.idservicio=s.idservicio
and s.idEspectacular=es.idEspectacular
and numeroEmpleado like concat(numemp,"%");
END #
\c 
DELIMITER ;
CALL trabajos(2);
\end{verbatim}


\subsection{DS-M: SQL Query \#12}
En esta consulta nos permite hacer la selección de los montos que el cliente tiene que pagar por el espectacular que esta rentando.
\begin{verbatim}
DELIMITER #
CREATE PROCEDURE pagos(in idespe int)
BEGIN
select e.idEspectacular, pdp.monto
from espectacular e, DetalleDeContrato ddc, Contrato c, ContratoTienePlanDePagos ctpdp, PlanDePagos pdp
where e.idEspectacular=ddc.idEspectacular
and ddc.idContrato=c.idContrato
and c.idContrato=ctpdp.idContrato
and ctpdp.idPlanDePagos=pdp.idPlanDePagos
and e.idEspectacular  like concat (idespe,"%");
END #
\c 
DELIMITER ;
CALL pagos(15);
\end{verbatim}


\subsection{DS-M: SQL Query \#13}
En esta consulta nos permite cargar un incidente que se halla suscitado con algún espectacular. Para poderle dar seguimiento al incidente. 
\begin{verbatim}
DELIMITER #
CREATE PROCEDURE incidentes(in idespe int, in idinci int,in clasi varchar(45),in descri varchar(200),

in feper DATE, in fehab DATE)
BEGIN
INSERT INTO Espectacular VALUES (idespe,"","","","","","","","","");
INSERT INTO Incidente VALUES (idinci,clasi,descri,feper,fehab,idespe);
END #
\c 
DELIMITER ;
CALL incidentes(idespe,idinci,clasi,descri,feper,fehab);

\end{verbatim}


\subsection{DS-M: SQL Query \#14}
En esta consulta nos permite ver todos los incidentes que ha tenido un espectacular en particular solo introduciendo el id del espectacular.
al hacer la consulta nos trae el id del espectacular y todos los detalles del incidente.
\begin{verbatim}
DELIMITER #
CREATE PROCEDURE selincidentes(in idespe int)
BEGIN
select e.idEspectacular, i.*
from espectacular e, incidente i
where e.idEspectacular=i.idEspectacular
and e.idEspectacular like concat (idespe"%"):
END #
\c 
DELIMITER ;
CALL selincidentes(15);
\end{verbatim}


\subsection{DS-M: SQL Query \#15}
En esta consulta no permite dar de alta a un empleado nuevo en la empresa para que pueda hacer uso del sistema facilitándole su tarea.
\begin{verbatim}
DELIMITER #
CREATE PROCEDURE altaempleados(in numemp int,in nomar varchar(30),in name varchar(45), 

in appat varchar(45), in apmat varchar(45), in emaill varchar(45), in tel varchar(15), in idar int)
BEGIN
INSERT INTO Empleado VALUES (numemp,name,appat,apmat,emaill,tel,"",idar,"","");
INSERT INTO Area VALUES (idar,nomar);
END #
\c 
DELIMITER ;
CALL altaempleados(numemp,nomar,name,appat,apmat,emaill,tel,idar);
\end{verbatim}


\subsection{DS-M: SQL Query \#16}
En esta consulta nos permite seleccionar a todos los empleados que están en la empresa y al área  la que pertenecen, con todos su datos.
\begin{verbatim}
DELIMITER #
CREATE PROCEDURE empleadostotal(in numemp int)
BEGIN
select e.*, a.*
from empleado e, area a
where e.idarea=a.idarea
and e.numemp like concat (numemp,"%");
END #
\c 
DELIMITER ;
CALL empleadostotal(18);
\end{verbatim}


\subsection{DS-M: SQL Query \#17}
En esta consulta nos permite dar de baja un espectacular del sistema para que ya no se pueda rentar después, la baja se da por varios casos explicados en el CU-03.
El dato necesario para dar de baja un espectacular sera por su id.
\begin{verbatim}
DELIMITER #
CREATE PROCEDURE eliminarespectacular(in idespe int)
BEGIN
DELETE 
FROM Espectacular 
WHERE idEspectacular=idespe;
END #
\c 
DELIMITER ;
CALL eliminarespectacular(18);
\end{verbatim}
\clearpage
\subsection{DS-M: SQL Query \#18}
Esta consulta nos permite hacer la modificación de algún espectacular ya sea en los campos del estado en el que se encuentra y el numero de personas a las que impactara.
\begin{verbatim}
DELIMITER #
CREATE PROCEDURE modificarespectacular(in idespe int,in desesta varchar(50), in impa int);
BEGIN
select e.idEspectacular, es.descripcionDeEstado, e.ImpactoEnPersonas 
from espectacular e, estado es
where e.idestado=es.idestado
and e.idEspectacular like concat (idespe,"%");
 
update espectacular e, estado es set es.descripcionDeEstado=desesta, e.ImpactoEnPersonas=impa
where e.idestado=es.idestado
and e.idEspectacular=idespe;

select e.idEspectacular, es.descripcionDeEstado, e.ImpactoEnPersonas 
from espectacular e, estado es
where e.idestado=es.idestado
and e.idEspectacular like concat (idespe,"%");
END #
\c 
DELIMITER ;
CALL modificarespectacular(idespe,desesta,impa);
\end{verbatim}

\subsection{DS-M: SQL Query \#19}
En esta consulta nos permite eliminar del sistema a un cliente que ya no usa nuestros servicios.
\begin{verbatim}
DELIMITER #
CREATE PROCEDURE eliminarcliente(in ideclie int)
BEGIN
DELETE 
FROM Cliente
WHERE idCliente=ideclie;
END #
\c 
DELIMITER ;
CALL eliminarcliente(18);
\end{verbatim}
\clearpage
\subsection{DS-M: SQL Query \#20}
En esta consulta nos permite hacer la modificación de la información de los clientes,la cual no es toda, únicamente es el domicilio del cliente, con todos sus parámetros.
\begin{verbatim}
DELIMITER #
CREATE PROCEDURE modificarcliente(in idclie int, in calleee varchar(60), in delegac varchar(60),

in cpp varchar(5), in colo varchar(45))
BEGIN
select *
from cliente
where idCliente like concat (idclie,"%");
 
update cliente set callee=calle, delegac=delegacion, cpp=cp, colo=colonia
where idCliente=idclie

select *
from cliente
where idCliente like concat (idclie,"%");
END #
\c 
DELIMITER ;
CALL modificarcliente(idclie,calleee,delegac,cpp,colo);
\end{verbatim}

\subsection{DS-M: SQL Query \#21}
En esta consulta nos permite eliminar del sistema a un empleado que ya no este laborando en la empresa para que ya no haga uso de su usuario para alterar de forma deliberada los registros del sistema.
\begin{verbatim}
DELIMITER #
CREATE PROCEDURE eliminarempleado(in numemple int)
BEGIN
DELETE 
FROM Espectacular 
WHERE numeroEmpleado=numemple;
END #
\c 
DELIMITER ;
CALL eliminarempleado(18);
\end{verbatim}
\clearpage
\subsection{DS-M: SQL Query \#22}
En esta consulta nos permite hacer la modificación de la información de los empleados,la cual no es toda, únicamente es el teléfono y si esta activo o no en la empresa.
\begin{verbatim}
DELIMITER #
CREATE PROCEDURE modificarempleado(in numemp int, in tel varchar(15),in act TINYINT)
BEGIN
select *
from empleado
where numeroEmpleado like concat (numemp,"%");
 
update empleado set numeroEmpleado=numemp, numeroTelefonico=tel, activo=act
where numeroEmpleado=numemp;

select *
from empleado
where numeroEmpleado like concat (numemp,"%");
END #
\c 
DELIMITER ;
CALL modificarempleado(numemp,tel,act);
\end{verbatim}
