\section{Propuesta de solución}
Con el fin de resolver los problemas anteriormente mencionados que enfrenta el IMSS, propone la creación de un software que permita mejorar y optimizar su funcionamiento haciendo que las distintas actividades que realizan los empleados día con día como son : El envío de correos, realización de llamadas, entre otras puedan ser llevadas a cabo de una manera más fácil, de forma organizada y eficiente.

\subsection{Objetivos generales}
Desarrollar un software para el IMSS que le permita tener un mayor control, organización e información relacionada con los empleados de las distintas Delegaciones y UMAES, así como mejorar la comunicación con los distintos departamentos para una centralización de los datos respecto a empleados, asimismo para evitar la pérdida de información y manteniéndola actualizada.

\subsection{Objetivos específicos}

\begin{itemize}
    \item Dar de alta a los nuevos empleados.
    \item Conocer la información de los empleados o de un sólo empleado de manera concreta.
    \item Modificar la información de algún empleado en específico.
    \item Dar de baja a empleados que ya no laboran dentro del instituto.
\end{itemize}

\subsection{Descripción de la solución}
Se propone un sistema el cuál contará con las siguientes características:\\
El \hyperlink{UsrDef:Administrador}{Administrador} podrá:
\begin{itemize}
	\item Tener una cuenta para acceder al sistema. 
	\item Dar de alta a nuevos empleados al directorio.
	\item Modificar la información de los empleados en caso de que sea necesario.
	\item Eliminar a un empleado en caso de que este ya no trabajé más para el instituto.
	\\
\end{itemize}

El \hyperlink{UsrDef:Consultor}{Consultor} podrá:
\begin{itemize}
	\item Visualizar la información de 
    \\
\end{itemize}

El sistema propuesto buscará concentrar la información de los empleados de una forma ordenada, facilitando a los demás empleados el acceso a la información de su interés. Beneficiando así la realización de las siguientes tareas:
\begin{itemize}
    \item Evitará la inconsistencia de información perteneciente a los empleados, como era el caso de el envío de información de un empleado laborando en el instituto a otro que dejó de laborar en el instituto.
\end{itemize}