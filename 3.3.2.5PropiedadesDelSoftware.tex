\subsubsection{Propiedades del software}
\begin{itemize}
    \item El software debe ser amigable con el cliente y los empleados que lo utilicen.
\item\textbf{Seguridad:} El usuario deberá proporcionar su nombre de usuario y contraseña para acceder al sistema.
\item\textbf{Usabilidad:} Nuestra meta es que todos los usuarios que utilicen el sistema aprendan a utilizarlo de forma correcta, así mismo que al menos el 95 por siento comprendan los resultados que se pueden obtener al utilizarlo. 
     \item \textbf{Eficiente:} El software debe ser eficiente, para esto se deben
cumplir las 3 subpropiedades: funcionalidad, tiempo de respuesta y utilización mınima de recursos.

\textbf{Eficiencia en tiempo de respuesta:}
Cuando el usuario realice una petición al sistema o seleccione una acción, nuestra meta es que el tiempo de respuesta (medido en segundos)  sea menor o igual a 5. 

\textbf{Funcionalidad:}
Se refiere a la capacidad de realizar el trabajo designado. 
Para comprobar si nuestro sistema es eficiente en funcionalidad,debemos de realizar una serie de pruebas basadas en los requerimientos funcionales que definimos con anterioridad y observar el funcionamiento del sistema.
 \item
\textbf{Atractivo} El software debe ser atractivo para el usuario.Para esto debemos acudir con el cliente y presentarle las diferentes opciones
acerca del diseño visual del sistema y compartir opiniones para llegar a un acuerdo y  desarrollar la versión final del
producto.
\item\textbf{Modificable:} Nuestra meta es que el sistema sea capaz de adoptar cualquier tipo de modificación o extensión de un modulo sin
dejar inservible algún otro. Para esto planeamos tomar el tiempo requerido para implementar una modificación en el sistema,
dicha modificación deberá incluir diseño, implementación y modificación de los cambios realizados.
\end{itemize}
\clearpage