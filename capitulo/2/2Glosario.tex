%=========================={Glosario}===========================
% En esta parte se pretende introducir los conceptos del negocio
%---------------------------------------------------------------

\chapter{Glosario}
\label{cap:Glosario}

\section{Definiciones}

\subsection*{Administrador}
\hypertarget{Glo:Administrador}
Es la persona o personas encargadas de realizar la alta, actualización o eliminación de la información de los distintos \hyperlink{Glo:Empleado}{Empleados}.

\subsection*{Consultor}
\hypertarget{Glo:Consultor}
Se define como la persona que realizará búsquedas dentro del \hyperlink{Glo:Directorio}{Directorio}.

\subsection*{CRUD}
\hypertarget{Glo:CRUD}
Es un acrónimo que se refiere a la acción de realizar altas, bajas y cambios en los datos que necesiten ocupar los \hyperlink{Glo:Empleado}{Empleados}, en este caso el término CRUD será usado para referirse a la interfaz gráfica de realizar altas, bajas y cambios que el \hyperlink{Glo:Editor}{Editor} con acceso al sistema requiera realizar a los datos.

\subsection*{Directorio}
\hypertarget{Glo:Directorio}
Es el almacén donde estará ubicada la información de los distintos \hyperlink{Glo:Empleado}{Empleados} dentro del \hyperlink{Glo:IMSS}{IMSS}.

\subsection*{Delegación}
\hypertarget{Glo:Delegación}
Se refiere a las Unidades Médicas de nivel 1 y 2 dentro del \hyperlink{Glo:IMSS}{IMSS}.

\subsection*{Editor}
\hypertarget{Glo:Editor}
Es la persona encargada de realizar únicamente la labor de actualización de la información de los distintos \hyperlink{Glo:Empleado}{Empleados}.

\subsection*{Empleado}
\hypertarget{Glo:Empleado}
Es la \hyperlink{Glo:PerFi}{Persona física} que labora dentro de las instalaciones del \hyperlink{Glo:IMSS}{IMSS} y que utiliza el sistema.

\subsection*{Identificador de usuario}
\hypertarget{Glo:Identificador de usuario}
Es aquella cadena de caracteres alfanuméricos que identifican de manera única a la cuenta del \hyperlink{Usuario}. 

\subsection*{IMSS}
\hypertarget{Glo:IMSS}
El Instituto Mexicano del Seguro Social o por sus siglas: IMSS. Es una Institución del gobierno federal, autónoma y tripartita.

\subsection*{UMAE}
\hypertarget{Glo:UMAE}
Es un acrónimo que se refiere a las Unidades Médicas de Alta Especialidad, el término UMAE se utilizará para para las Unidades Médicas de nivel 3.

\subsection*{Usuario}
\hypertarget{Glo:Usuario}
Un usuario son todos aquellos que interactúan con el sistema y cuentan con sus credenciales de acceso, que son su \hyperlink{Identificador de usuario} y su contraseña. Por cuestiones de seguridad, a cada usuario le corresponde una cuenta de usuario.

\newpage

\section{Definición de usuarios del sistema}

A continuación, se describen la importancia de los usuarios definidos en el glosario anterior, especificando sus responsabilidades entre ellos mismos y con el sistema al mismo tiempo que definimos su participación en distintos procesos.

\begin{Usuario}{\subsection{Administrador}}{
\hypertarget{UsrDef:Administrador}
	El administrador es el encargado de dar de alta, modificar o eliminar la información de los empleados para que los consultores al momento de necesitar la información se encuentre de manera correcta.\\
}
    \item[Responsabilidades:] \cdtEmpty
    \begin{itemize}
		\item Solicitar la información de los empleados dentro de las \hyperlink{Glo:Delegacion}{Delegaciones} y \hyperlink{Glo:UMAE}{UMAES} del \hyperlink{Glo:IMSS}{IMSS}.
		\item Dar de alta la información de los empleados dentro de las \hyperlink{Glo:Delegacion}{Delegaciones} y \hyperlink{Glo:UMAE}{UMAES} del \hyperlink{Glo:IMSS}{IMSS}.
    \end{itemize}

	\item[Procesos en los que participa:] \cdtEmpty
    \begin{itemize}
		\item Recabar información de los empleados.
		\item Modificar información de los empleados.
		\item Eliminar información de los empleados.
    \end{itemize}
\end{Usuario}

\begin{Usuario}{\subsection{Consultor}}{
\hypertarget{UsrDef:Consultor}
	El consultor es el encargado de visualizar la información de los empleados para posteriormente ponerse en contacto con ello s por algún otro medio de comunicación.\\
}
    \item[Responsabilidades:] \cdtEmpty
    \begin{itemize}
		\item Solicitar la modificación de la información de los empleados cuando es errónea.
		\item Consultar la información de los empleados de las distintas \hyperlink{Glo:Delegacion}{Delegaciones} y \hyperlink{Glo:UMAE}{UMAES} dentro del \hyperlink{Glo:IMSS}{IMSS}.
    \end{itemize}

	\item[Procesos en los que participa:] \cdtEmpty
    \begin{itemize}
		\item Recabar información a modificar de los empleados.
		\item Contactar al administrador o editor para pedirles la modificación de información de los empleados.
		\item Eliminar información de los empleados.
    \end{itemize}
\end{Usuario}

\begin{Usuario}{\subsection{Editor}}{
\hypertarget{UsrDef:Editor}
	El editor es el encargado de modificar o eliminar la información de los empleados para que los consultores al momento de necesitar la información se encuentre de manera correcta.\\
}
    \item[Responsabilidades:] \cdtEmpty
    \begin{itemize}
		\item Solicitar la información de los empleados dentro de las \hyperlink{Glo:Delegacion}{Delegaciones} y \hyperlink{Glo:UMAE}{UMAES} del \hyperlink{Glo:IMSS}{IMSS}.
		\item Dar de alta la información de los empleados dentro de las \hyperlink{Glo:Delegacion}{Delegaciones} y \hyperlink{Glo:UMAE}{UMAES} del \hyperlink{Glo:IMSS}{IMSS}.
		\item Eliminar la información de los empleados en caso de que sea requerido.
    \end{itemize}

	\item[Procesos en los que participa:] \cdtEmpty
    \begin{itemize}
        \item Recibir una notificación  o correo por parte del consultor para la solicitar la modificación de información de algún empleado.
		\item Recabar información a modificar de los empleados.
		\item Modificar información de los empleados.
		\item Eliminar información de los empleados.
    \end{itemize}
\end{Usuario}